\documentclass[times, utf8, zavrsni]{fer}
\usepackage{booktabs}
\usepackage{enumitem}

\begin{document}

% TODO: Navedite broj rada.
\thesisnumber{000}

% TODO: Navedite naslov rada.
\title{Baza podataka i web-aplikacija za plivačka natjecanja}

% TODO: Navedite vaše ime i prezime.
\author{Ilan Vezmarović}

\maketitle

% Ispis stranice s napomenom o umetanju izvornika rada. Uklonite naredbu \izvornik ako želite izbaciti tu stranicu.
% \izvornik

% Dodavanje zahvale ili prazne stranice. Ako ne želite dodati zahvalu, naredbu ostavite radi prazne stranice.
\zahvala{}

\tableofcontents

\chapter{Uvod}
U današnjem dobu digitalne revolucije, postoji sve veća potreba za praćenjem i analizom sportskih rezultata. 
Plivanje je jedan od sportova za koji je izražena potreba za pohranom velikog broja informacija.
Rezultat navedenog je velik broj sportskih aplikacija. Kod aplikacije vrlo bitno da bude
jednostavna i intuitivna za korisničko korištenje.

\vspace{\baselineskip}

Motivacija za izradu ove aplikacije proizlazi iz želje da se omogući plivačima, trenerima i 
drugim zainteresiranim korisnicima jednostavan i intuitivan pristup njihovim rezultatima. 
Kroz jednostavno korisničko sučelje, korisnici će moći brzo i lako pregledavati, 
analizirati i pratiti postignuća plivača na temelju plivačkih rezultata.

\vspace{\baselineskip}

Za izgradnju ovakvog sustava bitne su dvije stvari. Prvo se treba definirati
dobro strukturirana i skalabilna baza podataka u koju će se pohranjivati svi podaci.
Druga bitna stvar je oblikvanje preglednog i jednostavnog korisničkog sučelja koje će
korisnicima omogućiti ugodan pristup podacima.

\vspace{\baselineskip}

U prvom dijelu rada navedena je specifikacija zahtjeva koje sustav treba ispuniti i njihova analiza.
U drugom dijelu detaljno je opisano oblikavanje i struktura baze podataka. U trećem dijelu opisana je interna 
implementacija web aplikacije te upute za korisnika.

\chapter{Specifikacija zahtjeva}

\section{Analiza zahjteva}
Na početku izrade ovakvog sustava potrebno je definirati korisnike i njihove zahjteve koje će sustav omogućavati.
Primarni i jedini korisnik sustava je posjtetitelj koji pregledava podatke o plivačkim aktivnostima.

Osnovni zahtjev posjetitelja je pregled i pretraživanje podataka o:
\begin{itemize}
    \item[a)] plivačkim natjecanjima
    \item[b)] plivačkim rezultatima
    \item[c)] bazenima
    \item[d)] svjetskim rekordima
    \item[e)] limitima za nastupa na PH$^1$
    \item[f)] kvalificiranim plivačima za PH$^1$
    \item[g)] registriranim plivačima
\end{itemize}


Posjetitelj također ima opciju dodavanja novih natjecanja.


\footnote{Prvenstvo Hrvatske}

\section{Obrasci uporabe}

Modeliranje funkcionalnih zahtjeva sustava ostvareno je dijagramom obrazaca uporabe (Use case) izraženih kao UML dijagram
s tekstualnim opisima obrazaca. Dijagram opisuje tipičnu uporabu sustava značajnu korisniku (aktoru) te prikazuje pogled na sustav koji naglašava njegovo
vanjsko ponašanje prema korisniku (pogled interakcije). Na slici prikazan je dijagram obrazaca uporabe.

\section{Opis obrazaca uporabe}

\chapter{Baza podataka}
\section{Korištene tehnologije}
\section{ER Model}
\section{Relacijski model}
\section{Stvaranje baze podataka}

\chapter{Web aplikacija}
\section{Korištene tehnologije}
\section{Korisničke upute}


\chapter{Zaključak}
Zaključak.

\citation{oetiket2007lshort}

\bibliography{literatura}
\bibliographystyle{fer}

\begin{sazetak}
Sažetak na hrvatskom jeziku.

\kljucnerijeci{Ključne riječi, odvojene zarezima.}
\end{sazetak}

% TODO: Navedite naslov na engleskom jeziku.
\engtitle{Title}
\begin{abstract}
Abstract.

\keywords{Keywords.}
\end{abstract}

\end{document}
